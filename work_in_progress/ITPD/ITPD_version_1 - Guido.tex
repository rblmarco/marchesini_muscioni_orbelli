%% LyX 2.2.2 created this file.  For more info, see http://www.lyx.org/.
%% Do not edit unless you really know what you are doing.
\documentclass[english]{article}
\usepackage[T1]{fontenc}
\usepackage[latin9]{luainputenc}
\setlength{\parindent}{0bp}
\usepackage{color}
\usepackage{babel}
\usepackage{array}
\usepackage{float}
\usepackage{calc}
\usepackage{url}
\usepackage{graphicx}
\usepackage[unicode=true,pdfusetitle,
 bookmarks=true,bookmarksnumbered=false,bookmarksopen=false,
 breaklinks=false,pdfborder={0 0 0},pdfborderstyle={},backref=false,colorlinks=true]
 {hyperref}

\makeatletter

%%%%%%%%%%%%%%%%%%%%%%%%%%%%%% LyX specific LaTeX commands.
%% Because html converters don't know tabularnewline
\providecommand{\tabularnewline}{\\}

%%%%%%%%%%%%%%%%%%%%%%%%%%%%%% User specified LaTeX commands.
\usepackage{changepage}
\usepackage{geometry}

\makeatother

\begin{document}
\begin{figure}[t]
\centering{}\includegraphics[scale=0.4]{Images/front_page/Logo_polimi_grande}
\end{figure}


\title{{\huge{}Integration Test Plan Document}}

\author{\noindent Guido Muscioni (mat. 876151)\\
Marco Orbelli (mat. 876649)\\
Paola Marchesini (mat. 876541)}

\date{\phantom{x}}

\maketitle
\begin{figure}[H]
\begin{centering}
\includegraphics[scale=0.6]{Images/Logo}
\par\end{centering}
\caption{New brand logo.}

\end{figure}

\newpage{}

\tableofcontents{}\newpage{}

\section{Introduction}

\subsection{Purpose}

This document is written based on DD document, its aim is to define
the testing phase of the system. It also contains the description
of the tests about the components described in the Section 2 of the
document mention before. In particular, the tests will focus on the
components system interfaces in order to integrate all the functions
for the correct execution of the new PowerEnjoy system.

All the developers that take part in the testing and integration testing
phases have to read and keep in mind the contents of that document.
<TODO: verificare la correttezza di quest,ultima frase.> The reading
of that document is also suggested to all the developers of the systems,
as they can focus on the critical system components.

\subsubsection{Scope}

The scope of that system, is defined in the RASD document, then, here
there is a copy of the paragraph:

\hfill{}
\begin{center}
\fbox{\parbox[t]{0.7\columnwidth}{%
\textit{The aim of the project is to develop a digital system for
an electric car-sharing company, that is called PowerEnjoy. There
are not previous system so that document will describe the requirements
that deal with all the system that the customer wants.}

\textit{The users of PowerEnjoy system are:}
\begin{itemize}
\item \textit{Client;}
\item \textit{Employee.}
\end{itemize}
\textit{Both of these users are going to have unique credentials in
order to use the system, so they must be registered. }

\textit{The clients will use this system to reserve and use cars of
PowerEnjoy company.}

\textit{The employees will use that system in order to know what cars
need help.}%
}}
\par\end{center}

\subsection{List of Definitions and Abbreviations}

\label{subsec:Definitions}

<TODO:Completezza, definizioni delle percentuali fra scrittura del
codice e Junit test su quel codice.>

<TODO: general system controller>

\subsection{List of Reference Document}
\begin{itemize}
\item Specification document: Assignments AA 2016-2017;
\item Examples documents:
\begin{itemize}
\item Sample Integration Test Plan Document.
\end{itemize}
\item PayPal documentation (\url{https://developer.paypal.com/docs/} and
\url{https://www.paypal.com/it/webapps/mpp/merchant});
\item Jersey framework documentation (\url{https://jersey.java.net/documentation/latest/user-guide.html});
\item PhoneGap documentation (\url{http://docs.phonegap.com/});
\item General API documentation of mobile OS.
\end{itemize}
\newpage{}

\section{Integration Strategy}

\subsection{Entry Criteria}

\label{subsec:completeness}

In order to begin the integration testing phase, RASD and DD documents
must be approved by the company. Therefore these documents must not
be modified with addiotional considerable functionalities or with
relevant architectural changes. These facts are mentioned in order
to try to avoid further critical alterations which can result in architectural
degradationa (drift and erosion).

All what is mentioned before is used to have a complete view of the
interactions between PowerEnjoy components, following the top-down
methodology, used to build the Design Document.

Each component can be integrated only if it is, at least, 90\% complete.

The test phase will begin according to this completeness percentage,
defined in section \ref{subsec:Definitions}:
\begin{itemize}
\item 100\% for the completeness of the DataBase source, and 90\% for the
Data Access interface;
\item 80\% for the general system controller;
\item 80\% for the Acces Interface of the system.
\end{itemize}

\subsection{Elements to be Integrated}

\subsection{Integration Testing Strategy}

The testing phase will follow the bottom-up approach, however the
system is been designed keeping in mind a top-dowm view. In actual
fact the design decisions, took in the Design Document, have been
made in order to make the integration phase, and also integration
testing phase, easier <TODO: marco scrivila te>.

Following the bottom-up approach, once a component is complete, according
to the definition of completeness in section \ref{subsec:completeness},
it can be tested, using its own driver, imediately and independently
from other system components (the definition and decription of the
drivers can be find in section \ref{subsec:drivers}).

Using bottom-up pattern, the test will be make on the real components
that will be released with the first version of the system. Futhermore,
the bugs and the errors will be find as soon as possible in order
to reduce their propagation in the system. However, at the same time,
the top-down approach used in design phase make the result of the
system nicer than take into consideration only a general bottom-up
approach, which it will be result in a bit rough outcome.

In the integration phase, the tests of commercial components, that
have been used in the system, are not required, as it is supposed
that the producer companies have already tested them. However the
tests are required for the integration between PowerEnjoy system and
these type of components.

\subsection{Sequence of Component/Function Integration}

In this section it is described how the different components of our
system has been integrated. As a notation, it is used an arrow, which
mean that a component X (starting point) needs a component Y (head
of the arrow).

\subsubsection{Software Integration Sequence}

Following, it is explained in details how the bottom-up approch is
applied to the Power Enjoy system developed. In particular, further
the basic approch mentioned before, the components of the system are
also been grouped according to the functionalities they provide and
finally integrated together.

\paragraph{Data Access Interface}

The first elements that have to be integrated are the DataBase Source
and the Data Acces Interface. The DataBase, as mentioned before in
\label{ref:subsec-Definitions}, is complete, in order to test the
other components of the system using real data to perform queries
and to manage a complex system as close as possible to reality.

\paragraph{System Controller}

Once made the DB accessible, it is necessary to integrate external
Payment Gateway with the Payment Controller. In this way, when the
next component will be integrated with the Payment component, we can
test early their interaction.

>\textcompwordmark{}>IMMAGINE DEI DUE MODULI PAYMENT<\textcompwordmark{}<

The other integration, according to the functionalities criterion,
involved two important component of the Power Enjoy system: the Car
Controller and the Ride Controller. With the integration of this three
internal component, the system can manage a ride.

>\textcompwordmark{}>IMMAGINE DEI 3 COMPONENTI INSIEME<\textcompwordmark{}<

The third phase involved one of the most imposrtant component of the
Power Enjoy System: the List Controller. This particular component
manages the employees' list of car, keeping updated which cars need
attention. This component is integrateded with the Car Controller:
in this way it is guaranteed the maintenance of cars and the employees'
management.

>\textcompwordmark{}>IMMAGINE DEI DUE COMPONENTI<\textcompwordmark{}<

In the fourth phase the Sms Gateway and the Mail Gateway are integrated
with the Notification Dispatcher. After the integration of the notification
components, the Notification Dispatcher is integrated with the Reservation
Controller. This phase guarantees the integrations of the subcomponents
involved in the reservation of cars and in the comunications of the
neccessary informations to the client. 

>\textcompwordmark{}>IMMAGINE DEI 4 COMPONENTI INTEGRATI<\textcompwordmark{}<

Finally the Reservation Controller is integrated with the List Controller
and with the Car Controller, and all of the components mentioned before
plus the Account Controller are integrated With the dispatcher.

>\textcompwordmark{}>IMMAGINE SOLO DI TUTTI I CONTROLLER E DEL DISPATCHER<\textcompwordmark{}<

\subsubsection{Subsystem Integration Sequence}

>\textcompwordmark{}>SEZIONE PER NOI INUTILE IN QUANTO NON ABBIAMO
MAI DETTO DI AVERE SOTTOSISTEMI, LA TOGLIEREI E METTEREI LA SITUA
FINALE NELLA SEZIONE PRECEDENTE<\textcompwordmark{}<

\newpage{}

\section{Indivual Steps and Test Description}

\subsection{Account Controller}

\subsubsection*{createUser(allCredentials)}

\begin{table}[H]
\begin{centering}
\begin{tabular}{|>{\raggedright}p{5cm}|>{\raggedright}p{7cm}|}
\hline 
\textbf{Input} & \textbf{Effect}\tabularnewline
\hline 
\hline 
Some incomplete credentials. & A NullArgumentException is generated.\tabularnewline
\hline 
Unique credentials already exist in the database. & An InvalidArgumentException is generated.\tabularnewline
\hline 
Error in payment information, when they are used in order to withdraw
the symbolic payment amount. & An InvalidArgumentException is generated.\tabularnewline
\hline 
Valid input. & The user is inserted in the database. The password is generated randomly
in the DBMS and a mail with its is sent to the user. \tabularnewline
\hline 
\end{tabular}
\par\end{centering}
\caption{createUser test.}
\end{table}


\subsubsection*{checkLoginCredentials(loginCredentials)}

\begin{table}[H]
\begin{centering}
\begin{tabular}{|>{\raggedright}p{5cm}|>{\raggedright}p{7cm}|}
\hline 
\textbf{Input} & \textbf{Effect}\tabularnewline
\hline 
\hline 
Some incomplete login credentials. & A NullArgumentException is generated.\tabularnewline
\hline 
Some wrong login credentials. & An InvalidArgumentException is generated.\tabularnewline
\hline 
Valid input. & The system returns the user's personal page.\tabularnewline
\hline 
\end{tabular}
\par\end{centering}
\centering{}\caption{checkLoginCredentials test.}
\end{table}


\subsubsection*{retrieveAccountInformation(user)}

\begin{table}[H]
\begin{centering}
\begin{tabular}{|>{\raggedright}p{5cm}|>{\raggedright}p{7cm}|}
\hline 
\textbf{Input} & \textbf{Effect}\tabularnewline
\hline 
\hline 
Valid input. & The database returns the user's entry.\tabularnewline
\hline 
\end{tabular}
\par\end{centering}
\centering{}\caption{retrieveAccountInformation test.}
\end{table}


\subsubsection*{modify(user, parameter)}

\begin{table}[H]
\begin{centering}
\begin{tabular}{|>{\raggedright}p{5cm}|>{\raggedright}p{7cm}|}
\hline 
\textbf{Input} & \textbf{Effect}\tabularnewline
\hline 
\hline 
A null parameter. & A NullArgumentException is generated.\tabularnewline
\hline 
Unique parameter already exist in the database. & An InvalidArgumentException is generated.\tabularnewline
\hline 
Error in payment information, when they are used in order to withdraw
the symbolic payment amount. & An InvalidArgumentException is generated.\tabularnewline
\hline 
Valid input. & The database updates the user's entry with the parameter.\tabularnewline
\hline 
\end{tabular}
\par\end{centering}
\centering{}\caption{modify test.}
\end{table}


\subsection{Car Controller}

\subsubsection*{changePhaseOfCar(carId)}

\begin{table}[H]
\begin{centering}
\begin{tabular}{|>{\raggedright}p{5cm}|>{\raggedright}p{7cm}|}
\hline 
\textbf{Input} & \textbf{Effect}\tabularnewline
\hline 
\hline 
An invalid carId. & An InvalidArgumentException is generated.\tabularnewline
\hline 
Human error. & An InvalidOperationException is generated.\tabularnewline
\hline 
Valid request. & The database calculate the actual phase, and update the record of
the car.\tabularnewline
\hline 
\end{tabular}
\par\end{centering}
\caption{changePhaseOfCar test.}
\end{table}

<TODO: dire che � il caso in cui l'employee cambia stato ma lo stato
non va bene, e la query al database, seguendo le regiole definite
in alloy, restituisce uno statoi incompatibile. Riportiamo anche il
diagramma di codice di alloy con le definizioni.>

\subsubsection*{needHelp(stateOfCar)}

\begin{table}[H]
\begin{centering}
\begin{tabular}{|>{\raggedright}p{5cm}|>{\raggedright}p{7cm}|}
\hline 
\textbf{Input} & \textbf{Effect}\tabularnewline
\hline 
\hline 
No signal for call. & An InvalidOperationException is generated.\tabularnewline
\hline 
Valid request. & The system of the car call the call center.\tabularnewline
\hline 
\end{tabular}
\par\end{centering}
\centering{}\caption{needHelp test.}
\end{table}


\subsubsection*{closeTheCar(carId)}

\begin{table}[H]
\begin{centering}
\begin{tabular}{|>{\raggedright}p{5cm}|>{\raggedright}p{7cm}|}
\hline 
\textbf{Input} & \textbf{Effect}\tabularnewline
\hline 
\hline 
An invalid carId. & An InvalidArgumentException is generated.\tabularnewline
\hline 
Valid input. & The database calculate the actual phase, and update the record of
the car.\tabularnewline
\hline 
\end{tabular}
\par\end{centering}
\centering{}\caption{closeTheCar test.}
\end{table}

<TODO: dire che il sistema prende atto della posizione e nel caso
manda la multa e l'employee se � unsafe.>

\subsubsection*{openTheCar(carId)}

\begin{table}[H]
\begin{centering}
\begin{tabular}{|>{\raggedright}p{5cm}|>{\raggedright}p{7cm}|}
\hline 
\textbf{Input} & \textbf{Effect}\tabularnewline
\hline 
\hline 
An invalid carId. & An InvalidArgumentException is generated.\tabularnewline
\hline 
Valid input. & The system send the correct open message to the car and wait for its
response.\tabularnewline
\hline 
\end{tabular}
\par\end{centering}
\centering{}\caption{openTheCar test.}
\end{table}

<TODO: verificare Furthermore the database calculate the actual phase,
and update the record of the car.However if the response is negative
an FailedActionException.>

\subsection{List Controller}

\subsubsection*{createTypedListOfCar(listCreator)}

\begin{table}[H]
\begin{centering}
\begin{tabular}{|>{\raggedright}p{5cm}|>{\raggedright}p{7cm}|}
\hline 
\textbf{Input} & \textbf{Effect}\tabularnewline
\hline 
\hline 
A null parameter. & A NullArgumentException is raised.\tabularnewline
\hline 
Valid input. & The correct List is generated by the system, with a query on the database\tabularnewline
\hline 
\end{tabular}
\par\end{centering}
\caption{createTypedListOfCar test.}
\end{table}


\subsection{Reservation Controller}

\subsubsection*{createReservation(carId)}

\begin{table}[H]
\begin{centering}
\begin{tabular}{|>{\raggedright}p{5cm}|>{\raggedright}p{7cm}|}
\hline 
\textbf{Input} & \textbf{Effect}\tabularnewline
\hline 
\hline 
A null parameter. & A NullArgumentException is raised.\tabularnewline
\hline 
An invalid carId. & An InvalidArgumentException is generated.\tabularnewline
\hline 
Valid input. & A new entry is added to the reservation database table.\tabularnewline
\hline 
\end{tabular}
\par\end{centering}
\caption{createReservation test.}
\end{table}


\subsubsection*{deleteReservation(reservationId)}

\begin{table}[H]
\begin{centering}
\begin{tabular}{|>{\raggedright}p{5cm}|>{\raggedright}p{7cm}|}
\hline 
\textbf{Input} & \textbf{Effect}\tabularnewline
\hline 
\hline 
A null parameter. & A NullArgumentException is raised.\tabularnewline
\hline 
An invalid reservationId. & An InvalidArgumentException is generated.\tabularnewline
\hline 
Valid input. & The entry corresponding to the reservationId is modified.\tabularnewline
\hline 
\end{tabular}
\par\end{centering}
\centering{}\caption{deleteReservation test.}
\end{table}


\subsubsection*{sendReservationMessage(drivingLicense)}

\begin{table}[H]
\begin{centering}
\begin{tabular}{|>{\raggedright}p{5cm}|>{\raggedright}p{7cm}|}
\hline 
\textbf{Input} & \textbf{Effect}\tabularnewline
\hline 
\hline 
A null parameter. & A NullArgumentException is raised.\tabularnewline
\hline 
An invalid drivingLicense. & An InvalidArgumentException is generated.\tabularnewline
\hline 
Valid input. & The system use the SMS gateway in order to send the message.\tabularnewline
\hline 
\end{tabular}
\par\end{centering}
\centering{}\caption{sendReservationMessage test.}
\end{table}


\subsubsection*{getRemainigTime(reservationId)}

\begin{table}[H]
\begin{centering}
\begin{tabular}{|>{\raggedright}p{5cm}|>{\raggedright}p{7cm}|}
\hline 
\textbf{Input} & \textbf{Effect}\tabularnewline
\hline 
\hline 
A null parameter. & A NullArgumentException is raised.\tabularnewline
\hline 
An invalid reservationId. & An InvalidArgumentException is generated.\tabularnewline
\hline 
Valid input. & The system calculate the remaining by using a query on the database
with the reservationId.\tabularnewline
\hline 
\end{tabular}
\par\end{centering}
\centering{}\caption{getRemainingTime test.}
\end{table}


\subsubsection*{generatePath(positionOfTheUser, positionOfTheCar)}

\begin{table}[H]
\begin{centering}
\begin{tabular}{|>{\raggedright}p{5cm}|>{\raggedright}p{7cm}|}
\hline 
\textbf{Input} & \textbf{Effect}\tabularnewline
\hline 
\hline 
A null parameter. & A NullArgumentException is raised.\tabularnewline
\hline 
All parameters are null. & A NullArgumentException is raised.\tabularnewline
\hline 
An invalid reservationId. & An InvalidArgumentException is generated.\tabularnewline
\hline 
All parameters are invalid. & An InvalidArgumentException is generated.\tabularnewline
\hline 
Valid input. & The system sends the request with the two position using Google Map
API.\tabularnewline
\hline 
\end{tabular}
\par\end{centering}
\centering{}\caption{generatePath test.}
\end{table}


\subsection{Ride Controller}

\subsubsection*{finishTheRide(rideId)}

\begin{table}[H]
\begin{centering}
\begin{tabular}{|>{\raggedright}p{5cm}|>{\raggedright}p{7cm}|}
\hline 
\textbf{Input} & \textbf{Effect}\tabularnewline
\hline 
\hline 
A null rideId. & A NullArgumentException is raised.\tabularnewline
\hline 
An invalid rideId. & An InvalidArgumentException is generated.\tabularnewline
\hline 
Valid input. & The system updates the entry in the database\tabularnewline
\hline 
\end{tabular}
\par\end{centering}
\caption{finishTheRide test.}
\end{table}


\subsubsection*{computeFinalPrice(rideId)}

\begin{table}[H]
\begin{centering}
\begin{tabular}{|>{\raggedright}p{5cm}|>{\raggedright}p{7cm}|}
\hline 
\textbf{Input} & \textbf{Effect}\tabularnewline
\hline 
\hline 
A null rideId. & A NullArgumentException is raised.\tabularnewline
\hline 
An invalid rideId. & An InvalidArgumentException is generated.\tabularnewline
\hline 
Valid input. & The system calculate the final price with a query on the database
in order to retrieve the possible discounts.\tabularnewline
\hline 
\end{tabular}
\par\end{centering}
\caption{computeFinalPrice test.}
\end{table}


\subsubsection*{payTheRide(rideId)}

\begin{table}[H]
\begin{centering}
\begin{tabular}{|>{\raggedright}p{5cm}|>{\raggedright}p{7cm}|}
\hline 
\textbf{Input} & \textbf{Effect}\tabularnewline
\hline 
\hline 
A null rideId. & A NullArgumentException is raised.\tabularnewline
\hline 
An invalid rideId. & An InvalidArgumentException is generated.\tabularnewline
\hline 
Valid input. & The system send the correct payment information, taken from the database,
to the Payment gateway.\tabularnewline
\hline 
\end{tabular}
\par\end{centering}
\caption{payTheRide test.}
\end{table}

\newpage{}

\section{Performance Analysis}

During the integration test phase there must be present also a performance
analysis phase, in order to satisfy the non-functional requirements
defined in the RASD document. The main analysis mus be done on the
waiting time that the future users of the system, both employees and
normal users, have to wait in order to complete their request. In
the textbox reported below there are all the requirements mentioned
before.

\hfill{}
\begin{center}
\fbox{\parbox[t]{0.7\columnwidth}{%

\paragraph{\textit{Non-functional requirements}}
\begin{itemize}
\item \textit{When the user finishes his/her ride, he/she has 1 minute to
plug the car;}
\item \textit{The system closes the car 20 seconds after the user came out
from the vehicle;}
\item \textit{The system must answer to the user's requests within 10 seconds;}
\item \textit{The system must be available 24h/7d.}
\end{itemize}
%
}}
\par\end{center}

The availability of the system must also be tested. The two servers
contained in both the server farms, definend in DD document, must
be able to keep the system online also if one goes offline due to
possible errors or informatic attacks. <TODO:verify> In case of updates
of the system, they must be done separately on each server, so one
of the test cases has to check the availability of the system during
upgrades. 

To support the tests for the mobile application, thank to the use
of PhoneGap, the application can be tested only on one operating mobile
system. In actual fact the same scenarios of tests can be executed
on different platforms. In section \ref{sec:Tools} there are the
software that can be used in order to test an application builded
with PhoneGap.

In the RASD document are mentioned the minimum software requirements
that the mobile phone has to provide in order to run correctly the
PowerEnjoy application. That requirements are mapped into harware
requirements for each mobile OS, so the application must be done in
order to be executed with resonable time based on that hardware specification.
In the textbox there is a quick view of what we have just mentioned.
\begin{center}
\fbox{\parbox[t]{0.7\columnwidth}{%
\textit{To use the mobile application:}
\begin{itemize}
\item \textit{A smartphone with:}
\begin{itemize}
\item \textit{Android: KitKat (4.4) and later;}
\item \textit{iOS: version 6 or later;}
\item \textit{Windows Mobile: version 10 or later;}
\end{itemize}
\item \textit{An active number, with a plans that contains internet connection:}
\begin{itemize}
\item \textit{3G;}
\item \textit{4G: not required, but recommended. }
\end{itemize}
\item \textit{An amount of space in order to install the application.}
\end{itemize}
%
}}
\par\end{center}

To view the harware specification follow the link reported in table.
\begin{center}
\begin{tabular}{|>{\centering}p{5cm}|>{\centering}p{7cm}|}
\hline 
OS & Link to harware requirements\tabularnewline
\hline 
\hline 
Windows 10 Mobile & \url{https://msdn.microsoft.com/en-us/windows/hardware/commercialize/design/minimum/minimum-hardware-requirements-overview}\tabularnewline
\hline 
Android KitKat & \url{https://static.googleusercontent.com/media/source.android.com/it//compatibility/4.4/android-4.4-cdd.pdf}\tabularnewline
\hline 
IOS 6 & see the specification of the Iphone 4\tabularnewline
\hline 
\end{tabular}
\par\end{center}

The dimension of the final package of the application must theoretically
not exceed 50 MB.

Based on that specification, and on what is mentioned in this section,
the final mobile application must not exceed 100 MB of RAM on every
devices and the medium value of RAM occupied by the application has
to be about 80 MB.

\newpage{}

\section{Tools and Test Equipment Required}

\label{sec:Tools}

PhoneGap http://phonegap.com/blog/2015/10/27/testmunk-guest-post/

\newpage{}

\section{Program Stubs and Test Data Required}

\subsection{Program Stubs and Test Data}

\label{subsec:drivers}

\newpage{}

\section{Appendices}

\subsection{Used Tools}
\begin{itemize}
\item Microsoft Visio 2016: for all the diagrams in that document (as Testing
Diagrams, etc ...);
\item Lyx document processor: to write all the document;
\item SourceTree: used as GitHub manager;
\item GitHub: used to manage the shared building process of that document.
\end{itemize}
\newpage{}

\section{Hours of Work}

\begin{tabular}{|c|c|c|c|}
\hline 
Day & Guido Muscioni & Marco Orbelli & Paola Marchesini\tabularnewline
\hline 
\hline 
2/01 & 1 &  & \tabularnewline
\hline 
4/01 & 2 &  & \tabularnewline
\hline 
5/01 &  &  & \tabularnewline
\hline 
6/01 &  &  & \tabularnewline
\hline 
7/01 &  &  & \tabularnewline
\hline 
8/01 & 4 & 4 & 4\tabularnewline
\hline 
9/01 &  & 2 & \tabularnewline
\hline 
10/01 &  &  & \tabularnewline
\hline 
11/01 & 5 &  & 5\tabularnewline
\hline 
12/01 &  &  & \tabularnewline
\hline 
13/01 &  &  & \tabularnewline
\hline 
14/01 &  &  & \tabularnewline
\hline 
15/01 &  &  & \tabularnewline
\hline 
total &  &  & \tabularnewline
\hline 
\end{tabular}

\newpage{}

\listoffigures

\end{document}
